\section{Технико-экономическое обоснование }

\subsection{Характеристика разработанного по индивидуальному заказу
	программного средства}

Разработанное в дипломном проекте программное средство для голосового управления промышленным оборудованием с использованием технологий Vosk,С++ и одноплатного компьютера предназначено для использования заказчиком с целью автоматизации управления станками и производственными линиями без использования рук.

Основные функции программного продукта:
\begin{itemize}
	\item голосовое управление оборудованием:
	\begin{enumerate}[label=\alph*)]
		\item настройка ключевых слов и действий под конкретное оборудование;
		\item обработка команд в реальном времени;
	\end{enumerate}
	
	\item оффлайн-работа без зависимости от интернета;
	\item гибкость конфигурации:
	\begin{enumerate}[label=\alph*)]
		\item адаптация под различные типы станков;
		\item возможность добавления новых команд без перепрограммирования;
	\end{enumerate}
	
	\item повышение безопасности:
	\begin{enumerate}[label=\alph*)]
		\item локальная обработка данных (без передачи в облако);
		\item минимизация ошибок ввода за счет голосового контроля.
	\end{enumerate}
\end{itemize}

Основными заказчиками и пользователями являются:
\begin{itemize}
	\item малые и средние производственные предприятия;
	\item операторы станков;
	\item инженеры, нуждающиеся в голосовом управлении без отрыва от работы.
\end{itemize}


Применение программного продукта в производственно-хозяйственной деятельности обеспечит заказчику:
\begin{itemize}
	\item снижение времени простоя за счет быстрого голосового управления;
	\item уменьшение количества ошибок при вводе команд;
	\item повышение безопасности за счет отсутствия необходимости касаться панелей управления грязными руками;
	\item снижение затрат на автоматизацию по сравнению с промышленными SCADA-системами.
\end{itemize}


Основной потребностью в использовании программного продукта заказчиком является упрощение управления оборудованием и повышение эффективности производства. 

Аналогичные решения (например, Siemens Voice Control) требуют дорогостоящей интеграции, а облачные системы (Alexa, Google Assistant):
\begin{itemize}
	\item не работают оффлайн;
	\item не адаптированы для промышленных условий.
\end{itemize}




\subsection{Расчет затрат на разработку и цена программного средства, созданного по индивидуальному заказу} 


Для разработки программного средства для одноплатного компьютера необходимы следующие специалисты: программист на С++ и специалист по тестированию программного обеспечения. Трудозатраты на разработку программного средства у программиста на С++ будут равны 140 часам, в то время как у специалиста по тестированию программного обеспечения они будут равны 50 часам.

Расчет основной заработной платы разработчиков осуществляется по формуле:
\begin{equation}
	\mathrm{З_{о}} =  \mathrm{К_{пр}} \sum_{i=1}^n \text{З}_{\text{ч}i} \cdot t_i,
	\label{eq:formula}
\end{equation}

где $n$ - количество человек, задействованных в разработке голосового помощника; 
$\text{К}_{\text{пр}}$ - коэффициент премий; 
$\text{З}_{\text{ч}i}$ - часовая заработная плата $i$-го исполнителя (руб.); 
$t_i$ - трудоёмкость работ, выполняемых $i$-ым исполнителем.

Примем количество рабочих часов в месяце равным 168 часам. Расчет затрат на основную заработную плату разработчикам приведен в таблице~\ref{tab:salary}. 


\begin{table}[H]
	\caption{Расчет затрат на основную заработную плату команды разработчиков}
	\label{tab:salary}
	\centering 
	\begin{tblr}{
			width=\textwidth,
			colspec={*{5}{X[c]}},  % Все столбцы центрированы по умолчанию
			cell{4-6}{1-4} = {l},  % Объединенные ячейки (строки 4-6, столбцы 1-4) - по левому краю
			vlines,
		}
		\hline 
		Категория исполнителя & Месячный оклад, р. & Часовой оклад, р. & Трудоемкость работ, ч & Итого, р. \\ 
		\hline  
		Программист
		& 2400
		& 14,3
		& 140
		& 2002   \\
		\hline  
		Тестировщик & 1800  & 10,7  & 50  & 535  \\ 
		\hline   
		\SetCell[r=1,c=4]{11.4cm}{Итого}& & & & 2537 \\ 
		\hline   
		\SetCell[r=1,c=4]{11.4cm}{Премия и иные стимулирующие выплаты (50 \%)} & & & & 1268,5 \\ 
		\hline  
		\SetCell[r=1,c=4]{11.4cm}{Всего затрат на основную заработную плату разработчиков} & & & & 3805,5 \\ 
		
		\hline 
		
	\end{tblr}
	
\end{table}

Формирование цены программного средства на основе затрат приведено в таблице~\ref{tab:price-calculation}. 

\begin{table}[H]
	\caption{Формирования цены программного средства на основе затрат}
	\label{tab:price-calculation}
	\centering 
	\begin{tblr}{
			width=\textwidth,
			colspec={X[2,c]|X[3,c]|X[c]},  % [c] - центрирование для каждого столбца
			vlines,
		}
		\hline 
		Наименование статьи затрат  & Формула/таблица для расчёта & Значение, р. \\ 
		\hline  
		Основная заработная плата разработчиков & Таблица~\ref{tab:salary}  & 3805,5 \\
		\hline  
		Дополнительная заработная плата разработчиков  &
		
		\neqt{\text{З}_\text{д}=\frac{ \text{З}_\text{о}\cdot\text{Н}_\text{д}}{100}}
		где	$\text{Н}_\text{д}$ – норматив дополнительной заработной платы (10 \%)
		\eqt{{\text{З}}_{\text{д}} = \frac{3805{,}5 \cdot 10}{100\%}}
		
		
		& 380,55 \\ 
		\hline  
		Отчисления на социальные нужды   &
		
		\neqt{\text{Р}_\text{соц}=\frac{ (\text{З}_\text{о}+\text{З}_\text{д})\cdot\text{Н}_\text{соц}}{100}}
		где	$\text{Н}_\text{соц}$ – норматив отчислений от фонда оплаты труда (34,6 \%)
		\eqt{{\text{Р}}_{\text{соц}} = \frac{(3805{,}5 + 380,55) \cdot 34{,}6}{100}}
		
		
		& 1448,37 \\ 
		\hline  
		Прочие расходы    &
		
		\neqt{\text{Р}_\text{пр}=\frac{ \text{З}_\text{о}\cdot\text{Н}_\text{пр}}{100}}
		где	$\text{Н}_\text{пр}$ –  норматив прочих расходов (30 \%)
		\eqt{{\text{Р}}_{\text{пр}} = \frac{3805{,}5\cdot 30}{100}}
		
		
		& 1141,65 \\ 
		\hline
	\end{tblr}
	
\end{table}
\newpage

\noindent Продолжение таблицы \ref{tab:price-calculation}
\begin{center}
	\begin{tblr}{
		width=\textwidth,
		colspec={X[2,c]|X[3,c]|X[c]},  % [c] - центрирование для каждого столбца
		vlines,
	}
		\hline 
		Наименование статьи затрат  & Формула/таблица для расчёта & Значение, р. \\   
		\hline
		Общая сумма затрат на разработку    &
		
		\neqt{\text{З}_\text{р} = \text{З}_\text{о} + \text{З}_\text{д} + \text{Р}_\text{соц} + \text{Р}_\text{пр}}
		\eqt{
			\begin{aligned}
				\text{З}_\text{р} &= 3805{,}5 + 380{,}55 + 1448{,}37 \\ &\quad + 1141{,}65
			\end{aligned}
		}
		
		& 6 776,07 \\ 
		\hline 
		Плановая прибыль, включаемая в
		цену программного средства  & 
		
		\neqt{\mathrm{П_{п.c}} = \frac{\mathrm{З_{р}} \cdot \mathrm{Р_{п.с}}}{100}}			
		где Рп.с – рентабельность затрат на разработку программного средства (25\%)
			
		\eqt{\mathrm{П_{п.c}} = \frac{5748,59 \cdot 25}{100}}
		
		& 1149,72   \\
		\hline
		Отпускная цена программного
		средства & 
		\neqt{\mathrm{Ц_{п.c}} = {\mathrm{З_{р}} + \mathrm{П_{п.с}}}}
		\eqt{\mathrm{Ц_{п.c}} = {5748,58} + {1149,72}}
		& 6 898,3  \\
		\hline		
	\end{tblr}
\end{center}



В результате расчетов, приведенных в таблице  \ref{tab:price-calculation}, была получена отпускная цена программного средства, равная 8470,08 рублям. Цена получилась выше, чем предлагается фриланс-разработчиками в интернете по результатам поиска Google, и в то же время на порядок меньше, чем у более серьезных фирм-разработчиков. 


\subsection{Расчет результата от разработки и использования программного средства, созданного по индивидуальному заказу} 


Для организации-разработчика экономическим эффектом является прирост чистой прибыли, полученной от разработки и реализации программного средства заказчику.
Прибыль программного средства, реализованного организацией-разработчиком по отпускной цене, сформированной на основе затрат на разработку, рассчитывается по следующей формуле 5.8
\begin{equation}
	\Delta 	\mathrm{П_{ч}} = \mathrm{П_{п.с}} \left( 1 - \frac{	\mathrm{Н_{п}}}{100} \right),
\end{equation}
где Пп.с – прибыль, включаемая в цену программного средства, Нп –налог на прибыль. Так как компания является резидентом Парка высоких технологий, то оно освобождается от налога на прибыль. Поэтому получим соответствующее значение:
\[
\Delta \mathrm{П_{ч}} = 1149.72 \left(1 - \frac{0}{100}\right) = 1149.72 \text{ р}.
\]

\subsection{Расчет показателей экономической эффективности разработки и использования программного средства} 

Для организации-разработчика программного средства оценка экономической эффективности разработки осуществляется с помощью расчета простой нормы прибыли (рентабельности инвестиций (затрат) на разработку программного средства) по формуле:
\begin{equation}
	\mathrm{Р_{и}} = \frac{\Delta 	\mathrm{П_{ч}}}{\mathrm{З_{р}}} \cdot 100\%,
\end{equation}
Подставив значения, посчитанные в разделах 5.2 и 5.3 по формулам 5.5 и 5.10, в формулу, получим следующий результат:
\[
\mathrm{Р_{и}} = \frac{1149.72}{5748.585} \cdot 100\% = 20\%.
\]
Для расчёта показателей экономической эффективности разработки и использования приложения необходимо полученные суммы результата (чистой прибыли) и затрат (инвестиций в разработку программного средства) по годам привести к единому моменту времени – расчётному году (2025 г.) путём умножения результатов и затрат за каждый год на коэффициент дисконтирования(αt) который рассчитывается по формуле:
\begin{equation}
	\alpha_t = \frac{1}{(1 + d)^{t - t_p}},
\end{equation}
где d – требуемая норма дисконта, 15\%;

t – номер года, результаты и затраты которого приводятся к расчётному (2025 – 1, 2026 – 2, 2027 – 3).

\[
\alpha_1 = \frac{1}{(1+0{,}15)^0} = 1,
\]
\[
\alpha_2 = \frac{1}{(1+0{,}15)^1} = 0{,}87,
\]

\[
\alpha_3 = \frac{1}{(1+0{,}15)^2} = 0{,}76
\]

Расчёт рентабельности инвестиций без учёта фактора времени производится по формуле:
\begin{equation}
	\mathrm{Р_{и}} = \frac{\mathrm{П_{чср}}}{\sum_{t=1}^{n} \text{з}_t} \cdot 100\%,
\end{equation}
где Пчср – среднегодовая чистая прибыль, полученной от использования разработанного программного средства которая определяется по формуле:
\begin{equation}
	\Pi_{\mathrm{upp}} = \frac{\sum_{i=1}^{n} \Pi_{\mathrm{ut}}}{n},
\end{equation}
\newpage